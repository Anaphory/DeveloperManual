%%%%%%%%%%%%%%%%%%%%%%%%%%%%%%%%%%%%%%%%%
% Journal Article
% LaTeX Template
% Version 1.4 (15/5/16)
%
% This template has been downloaded from:
% http://www.LaTeXTemplates.com
%
% Original author:
% Frits Wenneker (http://www.howtotex.com) with extensive modifications by
% Vel (vel@LaTeXTemplates.com) and Fabio Mendes (f.mendes@auckland.ac.nz)
%
% License:
% CC BY-NC-SA 3.0 (http://creativecommons.org/licenses/by-nc-sa/3.0/)
%
%%%%%%%%%%%%%%%%%%%%%%%%%%%%%%%%%%%%%%%%%

%----------------------------------------------------------------------------------------
%	PACKAGES AND OTHER DOCUMENT CONFIGURATIONS
%----------------------------------------------------------------------------------------

%\documentclass[oneside,twocolumn]{article}
\documentclass[oneside]{article}

\usepackage{blindtext} % Package to generate dummy text throughout this template 

\usepackage{graphicx} % FKM: for figures
\usepackage{color} % FKM: for colored text
\usepackage{float} % FKM: for forcing figure placement
\usepackage[breakable]{tcolorbox} % FKM: for text box
\usepackage{enumerate} % FKM: for bullet point lists
\usepackage{setspace} % FKM: line spacing
\usepackage[margin=1in]{geometry} % FKM: margins
\usepackage{listings} % FKM: for code windows (lstlisting)
\usepackage{tikz} % FKM: using for \checkmark
\def\checkmark{\tikz\fill[scale=0.4](0,.35) -- (.25,0) -- (1,.7) -- (.25,.15) -- cycle;}

\usepackage[sc]{mathpazo} % Use the Palatino font
\usepackage[T1]{fontenc} % Use 8-bit encoding that has 256 glyphs
% \linespread{1.05} % Line spacing - Palatino needs more space between lines
\usepackage{microtype} % Slightly tweak font spacing for aestheticsgins
\usepackage[small,labelfont=bf,up,up]{caption} % Custom captions under/above floats in tables or figures
\usepackage{booktabs} % Horizontal rules in tables
\usepackage{amsmath} % Text in equations

\usepackage[shortlabels]{enumitem} % customized lists (shortlabels
                                % necessary to have i., ii., etc., in enumerate)
\setlist[itemize]{noitemsep} % Make itemize lists more compact

\usepackage{abstract} % Allows abstract customization
\renewcommand{\abstractnamefont}{\normalfont\bfseries} % Set the "Abstract" text to bold
\renewcommand{\abstracttextfont}{\normalfont\small\itshape} % Set the abstract itself to small italic text

\usepackage{titlesec} % Allows customization of titles

\usepackage{fancyhdr} % Headers and footers
\pagestyle{fancy} % All pages have headers and footers
\fancyhead{} % Blank out the default header
\fancyfoot{} % Blank out the default footer
\fancyhead[C]{Authors et al. $\bullet$ August 2018 $\bullet$ bio{\color{red}R}$\chi$ve} % Custom header text
\fancyfoot[RO,LE]{\thepage} % Custom footer text

\usepackage{titling} % Customizing the title section

\usepackage[hidelinks]{hyperref} % For hyperlinks in the PDF

\usepackage{natbib}
\bibliographystyle{apalike}

\setlength\columnsep{20pt}

\definecolor{pblue}{rgb}{0.13,0.13,1}
\definecolor{pgreen}{rgb}{0,0.5,0}
\definecolor{pred}{rgb}{0.9,0,0}
\definecolor{pgrey}{rgb}{0.46,0.45,0.48}
\lstset{language=Java,
  showspaces=false,
  showtabs=false,
  breaklines=true,
  showstringspaces=false,
  breakatwhitespace=true,
  commentstyle=\color{pgreen},
  keywordstyle=\color{pblue},
  stringstyle=\color{pred},
  basicstyle=\ttfamily,
  moredelim=[il][\textcolor{pgrey}]{$$},
  moredelim=[is][\textcolor{pgrey}]{\%\%}{\%\%}
}
%----------------------------------------------------------------------------------------
%	TITLE SECTION
%----------------------------------------------------------------------------------------

\setlength{\droptitle}{-4\baselineskip} % Move the title up

\pretitle{\begin{center}\Huge\bfseries} % Article title formatting
\posttitle{\end{center}} % Article title closing formatting
\title{{\huge SUPPLEMENTARY MATERIAL}\\\vspace{.5cm}How to validate your probabilistic model
  implementation} % Article title
\author{\textsc{An author here$^{1,2*}$}, \textsc{Another author
    here$^{1*}$}, \\ \textsc{Yet another author here$^{1*}$},
  \textsc{Last author here$^{1*}$} \\
\small $^1$School of Computer Science, The University of Auckland\\
\small $^2$School of Biological Sciences, The University of Auckland\\
\small
\href{mailto:f.mendes@auckland.ac.nz}{Corresponding authors$^*$: f.mendes@auckland.ac.nz,}
\href{mailto:f.mendes@auckland.ac.nz}{another.email@auckland.ac.nz}
%\and % Uncomment if 2 authors are required, duplicate these 4 lines if more
%\textsc{Jane Smith}\thanks{Corresponding author} \\[1ex] % Second author's name
%\normalsize University of Utah \\ % Second author's institution
%\normalsize \href{mailto:jane@smith.com}{jane@smith.com} % Second author's email address
}
\date{\today} % Leave empty to omit a date
% \renewcommand{\maketitlehookd}{}

%----------------------------------------------------------------------------------------

\doublespacing

\begin{document}

% Print the title
\maketitle

\clearpage
%----------------------------------------------------------------------------------------
%	ARTICLE CONTENTS
%----------------------------------------------------------------------------------------

\section{Validation procedures in the literature}

The literature on probabilistic modelling in evolutionary biology is vast,
and a thorough, exhaustive review of this literature is outside the
scope of the present work. 
We do however provide a brief, non-exhaustive list of references that
have implemented (and potentially validated) phylogenetic Bayesian
models (Table \ref{tab:papers}) and other inference components such as
operators.
Each reference was annotated for different validation categories (these
are discussed in the main text).
Our goal was to capture the variation in validation stringency and
comprehensiveness present in this set of computational tools.

\begin{center}
  \begin{table}[H]
  \caption{A non-exhaustive list of references from the Bayesian
    phylogenetics literature and the validation procedures explicitly
    mentioned in the main or supplementary text.}
  \label{tab:papers}
  \centering
  \begin{tabular}{ c|c|c|c|c }
    \hline
    Reference & i & ii & iii & (a) \\
    \hline  
    \citealp{mau99} & & \checkmark & & \\
    \citealp{huelsenbeck00} & & & & \\
    \citealp{drummond02} & \checkmark & & & \checkmark \\
    \citealp{eastman11} & & & & \checkmark \\
    \citealp{uyeda14} & & \checkmark & & \\
    \citealp{kostikova16} & & \checkmark & & \\
    \citealp{caetano17} & & & & \\
    \citealp{carretero18} & & & & \\
    \hline
  \end{tabular}
  \end{table}
\end{center}

A reference was annotated as having conducted a validation
procedure if it explicitly mentioned the procedure in the main or
supplementary text.
We note that we list references implementing models from the ground
up, meaning that ideally some validation should have been
reported in the manuscript.
It is perfectly possible, however, that some of these procedures were in fact
carried out prior to or during publication, but were simply not
mentioned in the text.
We would hope that unit tests for the likelihood function (item ``i''
in Tab. \ref{tab:papers}), for example, have been implemented in most
if not all of the software we listed.
Alternatively, a reference might belong to a group of several
publications collectively documenting the development of a large
software package, in which case previous papers could have documented 
the validation of at least part of the Bayesian inference machinery,
such as MCMC operators.

\section*{Validation examples}
Below we provide code examples for the validation procedures mentioned
in the main text.
The code excerpts are written in Java or xml, as they use BEAST 2
model implementations as the reference software being validated.

\subsection*{Computing the likelihood and comparing against an expected value}

The simplest validation test for an implementation of a likelihood
function is to evaluate the likelihood of a parameter value given some
data.
Ideally, the expected value will have been analytically derived from the
model under simple conditions (e.g., a small phylogenetic tree,
equating some of the parameters to zero or unity, etc.).
This forms the basis of what software engineers refer to as ``unit testing''.
We show a unit test example for the Yule model in listing \ref{lst:yuleunit}.

\singlespacing
{\small
\begin{lstlisting}[language=Java, caption=Unit test for Yule model
  likelihood function given small phylogenetic tree.,label={lst:yuleunit}]
  
package test;

import org.junit.Test;
import beast.evolution.speciation.YuleModel;
import beast.util.TreeParser;
import junit.framework.TestCase;

public class YuleLikelihoodTest extends TestCase {
    @Test
    public void testYuleLikelihood() {
        TreeParser tree = new TreeParser("((A:1.0,B:1.0):1.0,(C:1.0,D:1.0):1.0);");
        YuleModel likelihood = new YuleModel();
        likelihood.initByName("tree", tree, "birthDiffRate", "1.0");

        assertEquals(-6.0, likelihood.calculateLogP());
   }
 }
\end{lstlisting}
}

%----------------------------------------------------------------------------------------
%	REFERENCE LIST
%----------------------------------------------------------------------------------------

\section*{References}
\clearpage

\bibliography{refs}

%----------------------------------------------------------------------------------------

\end{document}
